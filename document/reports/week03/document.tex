\documentclass[preprint]{aastex62}
% \documentclass[manuscript]{aastex62}
%%  twocolumn, manuscript, preprint, preprint, modern and RNAAS
\usepackage[utf8]{inputenc}
\usepackage{lineno}
\newcommand{\vdag}{(v)^\dagger}
\newcommand\aastex{AAS\TeX}
\newcommand\latex{La\TeX}
%% Tells LaTeX to search for image files in the 
%% current directory as well as in the figures/ folder.
\graphicspath{{./}{figures/}}
%% Reintroduced the \received and \accepted commands from AASTeX v5.2
%%\received{January 1, 2018}
%%\revised{January 7, 2018}
%%\accepted{\today}
\shorttitle{Weekly Activity Report}
\shortauthors{Felipe L. G\'omez-Cort\'es}

%\renewcommand{\abstractname}{RESUMEN}
%\renewcommand{\figurename}{Figura}

\linenumbers
\begin{document}

\title{Weekly Activity Report\\Week 3\\A second algorithm using a random point catalog\\and the 1-Skeleton}

\correspondingauthor{Felipe Leonardo Gómez-Cortés}
\email{fl.gomez10@uniandes.edu.co}

\author{Felipe Leonardo Gómez-Cortés}
\affiliation{Physics Department, Universidad de Los Andes}
\collaboration{Master Student}

\nocollaboration

\author{Jaime E. Forero-Romero}
\affiliation{Physics Department, Universidad de Los Andes}
\collaboration{Advisor}


\begin{abstract}
  The new algorithm relies on a random set of points. The $\beta$-Skeleton is build
  over the junction set of points (observations plus random points).
  
  \keywords{ Beta Skeleton Graph, Random set of points}

\end{abstract}

\section{The Algorithm}

The main goal of this project is to identify voids in the large scale structure
of the universe, using the nobel method in astrophysics: the $\beta$-Skeleton
graph. In order to develop the code, a toy model catalog structure is used,
then a second catalog of random points populates the same volume, the graph is
calculated and then the voids are identified as the second catalog points
without connections with the first catalog points.

\begin{figure}
    \plotone{slice_OC_plus_RC.pdf}
    \caption{The Algorithm uses two sets of points: The Observed Catalog
      (LSS like in blue with a spherical void in the middle) and the
      random Catalog (orange). \label{slice_OC_plus_RC}}
\end{figure}

We start with a sample of N points randomly placed inside a cubic box of lenght L. (Figure \ref{slice_OC_plus_RC}.
In the middle of the box there is a spherical void, i.e. an spherical region
without points. This set of points is called ``the observed catalog''
(\textbf{OC}). This toy model
will represent -in a gross aproximation- the LSS with galaxies and voids.

Then another set of N random points populates the whole box, without restrictions.
This set is called ``the random catalog'' (\textbf{RC}). It will populate even
the void region. It's only necesary to know the volume of the OC and the number
of points N to create the RC.

The NGRAPH library can operate over a single set of points. This set is called
``the full catalog'' (\textbf{FC}). This set of points is created using the
\texttt{vstack} python function. The first N elements are the RC points, the last
N elements are the OC points, then the FC has 2N elements.

The 1-Skeleton graph is calculated over the FC. Runing the code over a FC of
$\sim 10^4$ points in a Core i5 (2nd gen.) Linux machine, it takes 192 seconds
to complete the calculus. The result is stored as the Full Catalog Beta
Skeleton Graph (\textbf{fcBSk}).

The fcBSk is chopped to the half, we are interested about the RC points that
are not connected with the OC points. Then a droplist is created as the
subset of connections where the second point belongs to the OC, i.e., its index
is greater than N.

Using the python  \texttt{set} class, its easy to find the complement of the
droplist in the fsBSk. After droping the RC points connected with the OC, it
remains the points inside the void.


\begin{figure}
    \plotone{slice_OC_RC_VoidPoints.pdf}
    \caption{The Algorithm is currently under development. It fails
      to recognize void points (Green).
      \label{slice_OC_RC_VoidPoints}}
\end{figure}


\section{Conclusions}

This new algorithm looks promising. Must be fully developed.

  \nocite{*}

  \begin{thebibliography}{20}

    %\bibitem[Bos et al.(2012)]{Bos2012} Bos, P. et al. \ 2012, \mnras, 426, 440 % Testing cosmologies using void ellipticity

  \end{thebibliography}                                                           
                       

%\listofchanges

\end{document}
