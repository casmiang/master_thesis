\documentclass[preprint]{aastex62}
\usepackage[utf8]{inputenc}
\usepackage{lineno}
\usepackage{multirow}
\newcommand{\vdag}{(v)^\dagger}
\newcommand\aastex{AAS\TeX}
\newcommand\latex{La\TeX}
\graphicspath{{./}{figures/}}
\shorttitle{Weekly Activity Report}
\shortauthors{Felipe L. G\'omez-Cort\'es}

\linenumbers
\begin{document}

\title{A Large Scale Structure Void Identifier for Galaxy Surveys based on the $\beta$-Skeleton Graph}

\correspondingauthor{Felipe L. Gómez-Cortés}
\email{fl.gomez10@uniandes.edu.co}

\author{Felipe Leonardo Gómez-Cortés}
\affiliation{Physics Department, Universidad de Los Andes, Bogotá, 111711, Colombia}

\author{Jaime E. Forero-Romero}
\affiliation{Physics Department, Universidad de Los Andes, Bogotá, 111711, Colombia}

\author{Xiao-Dong Li}
\affiliation{School of Physics and Astronomy, Sun Yat-Sen University, Guangzhou 510297, P. R. China}

\section{The Algorithm}

Our approach solving the problem of finding voids in the cosmic web is from
a geometrical point of view. We use the $\beta$-Skeleton graph to generate
the connections between a big set of points belonging to $\mathbb{R}^3$
in comoving coordinates. This collection consists of two subsets: LSS points
and Random Points.

We call the subset of Large Scale Structure tracers as The Observed Catalog
(\textbf{OC}). They may come from observed galaxies (\textit{e.g.} SDSS)
or dark matter halo catalogs from N-body simulations (\textit{e.g.} Abacus
Cosmos). The OC shows the well-known filamentary structure tha can be found
easily by the $\beta$-Skeleton libraries. The number of OC points is named
$N_{OC}$.

The conterpart is the random set of points. They uniformly fill the space,
its $\beta$-skeleton shows that the mean number of connections different
from the filamentary structure mean connections. This Random Catalog
(\textbf{RC}) is generated using an uniform PDF. The number of RC points
is named $N_{RC}$.

The Full Catalog (\textbf{FC}) of points is created merging the observed
catalog and the random catalog. The random points fill the cosmic voids.
Then we calculate the FC's $\beta$-skeleton. We find random points connected
to the LSS tracers near and into the filaments. But we find random points
connected only to random points exclusively inside voids, they can be
named ``pure void points''.

The algorithm uses the FC's $\beta$-Skeleton to discard random points
connected to LSS tracers. This first filtered $\beta$-Skeleton graph is
used by the algoritm to identify the first connections of pure void points
and stores the list of results in a first particle void list.
The algorithm checks element by element if pure voids are connected or
share connections, if the condition is true, then the two sets are merged.
At the ending we have a set of voids. Each void is a list of points
connected between them. Two voids never are connected.

\subsection{Calibrating our free parameters}
How to select $\beta$ and the ratio $N_{rand}/N_{obs}$

\subsection{Measuring Void Size}

A first approach is to asign a volume proportional to the number of particles
in the void times the density (total number of random particles over total
catalog volume.

A second method asumes the voids as ellipsoid objects. 
We use the standard equation for the surface of an ellipsoid:
\begin{equation}
\frac{x^2}{a^2} + \frac{y^2}{b^2} + \frac{z^2}{c^2} = 1 
\end{equation}
so, $a$, $b$ and $c$ are the semi-axes of the ellipsoid. The volume enclosed is:
$V = \frac{4\pi}{3}abc$.
Semi-axes are found using the inertia tensor of the point distribution.
\begin{equation}
    I = \frac{m}{5} \left( 
    \begin{array}{ccc}
    b^2 + c^2 &  0  &  0 \\
    0   & a^2 + c^2 &  0 \\ 
    0   &  0  &  a^2 + b^2 
    \end{array}\right)
\end{equation}

Numpy's \textit{linalg} module calcules the eigenvalues. It
returns
\begin{equation}
    I = \left( 
    \begin{array}{ccc}
    I_a &  0  &  0 \\
    0   & I_b &  0 \\ 
    0   &  0  &  I_c 
    \end{array}\right)
\end{equation}
and the rotation matrix (eigenvectors).

Asuming $m=1$, or giving a mass $m_i = 1/N_{void}$ for each one of the $N_{void}$
particles of the void, we can calculate the void semi-axes independently from the
number of particles in the void.

\begin{eqnarray}
  a &=& \sqrt{ \frac{5}{2N} \left(-I_a + I_b + I_c \right)} \nonumber\\
  b &=& \sqrt{ \frac{5}{2N} \left( I_a - I_b + I_c \right)} \nonumber\\
  c &=& \sqrt{ \frac{5}{2N} \left( I_a + I_b - I_c \right)} \nonumber\\
\end{eqnarray}

They may not be sort as $a>b>c$, but this is solved in a few code lines.


\subsection{Void Abundance}

\end{document}
