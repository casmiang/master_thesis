\documentclass[preprint]{aastex62}
\usepackage[utf8]{inputenc}
\usepackage{lineno}
\usepackage{multirow}
\newcommand{\vdag}{(v)^\dagger}
\newcommand\aastex{AAS\TeX}
\newcommand\latex{La\TeX}
\graphicspath{{./}{figures/}}
\shorttitle{Weekly Activity Report}
\shortauthors{Felipe L. G\'omez-Cort\'es}

\linenumbers
\begin{document}

\title{Weekly Activity Report\\Week 6\\The Xiao-Dong Li's $\beta$-Skeleton Finder}

\correspondingauthor{Felipe Leonardo Gómez-Cortés}
\email{fl.gomez10@uniandes.edu.co}

\author{Felipe Leonardo Gómez-Cortés}
\affiliation{Physics Department, Universidad de Los Andes}
\collaboration{Master Student}

\nocollaboration

\author{Jaime E. Forero-Romero}
\affiliation{Physics Department, Universidad de Los Andes}
\collaboration{Advisor}


\begin{abstract}
  Xiao-Dong Li developed a series of codes to run over LSS data. I used part of the code
  available from his repo at github. The code was installed at the Magnus Cluster.
  
\end{abstract}

\section{LSS Codes}
Xiao-Dong Li cerated in 2015 a big Fortran90 library whith tools to study the LSS.
The $\beta$-Skeleton finder is part of the library, as well two and three point
correlation functions, redshift corrections, lightcone generators and some
statistical tools to run over mocks.

This library was created using the ifort compiler (Intel-Fortran). This is privative
software, not available at the Magnus Cluster (the Uniandes HPC facility). Ifort
has been developed to run 5-10 times faster than gfortran over Intel-Xeon
processores.

During the installation over the Magnus Cluster some errors rise. It was necessary
to modify the original code to compile using gfortran.

This fortran code runs in the cluster near to $\sim 100-1000$ times faster than the NGL
library on my laptop. XDL code divides the space in cells to search closest neighbours
instead searching over the whole set of points. This search has the parameter
``neighbours'' to reduce the number of close points to make the search. The code can
switch between exact and non-exact calculaton of nearest neighbours for the
$\beta$-skeleton graph. It gives some differences in the final graph.

In the table \ref{tab:running_timeNGLvsXDL}
is possible to see how an threshold value of the
neighbour parameter gives different running times and number of connections.
The author suggests a value around 100 neighbours to make calculations for the
fixed value $\beta=1$. But with valuse from 200 neighbours and above the final
$\beta$-graph is identical. The files marked with asterisks where checked with the
md5sum algorithm.

The output file from booth algorithms differs. NGL shows the full list of connections
for each one of the N points. It means that each connection is counted a twice.
E.g. If the points 15 and 99 are connected, in the output file appears (15,99) and
(99,15). XDL avoids to count repeated pairs, following the example, only (15,99) will
appear on the output file. It gives a different size of output files.

\section{Comparing NGL and XDL $\beta$-Graphs}

A mandatory check when a novel method is implemented is to compare against
previous methods and results.  I was focused on the 2N:N dataset (N = 9998
observed points from Abacus Simulations), having a total number of points
of $\sim 30000$.

Setting the shearching parameter in 100 neighbours throws the same number
of connections as in XDL as in NGL graphs. But a detailed search shows that 
there are four points missmatching: one pair connection that appears on NGL
but not in XDL (indices 1878-13682) and another extra connection on XDL
missing in NGL (indices 6765-12474).

Setting the searching parameter in 200 and above throws one extra connection
in the XDL graph respect to the NGL graph. (Is the same connection of indices
6752-12474). It means one extra connection in 110225. Plotting the data it
seems a legal one (figure \ref{fig:extra_connection}).

\startlongtable
\begin{deluxetable}{c|cc|ccc}
  \tablecaption{Running Time of $\beta$-Skeleton using NGL and XDL libraries.
    \label{tab:running_timeNGLvsXDL}}
\tablehead{
  \colhead{Number} & \multicolumn{2}{c}{NGL} & \multicolumn{3}{c}{Xiao-Dong Li's Library}\\
  \colhead{Particles}
  & \colhead{Time (s)} & \colhead{Connections}
  & \colhead{Time (s)} & \colhead{Neighbours} & \colhead{Connections}
}%\colnumbers
\startdata
\multirow{5}[0]{*}{20000}&
\multirow{5}[0]{*}{1717} &
\multirow{5}[0]{*}{139504}&1.4&   50 &  69427 \\
      &       &        & 2.4  &  100 &  69749 \\
      &       &        & 6.4  &  300 &  69752 \\
      &       &        & 11   &  600 &  69752 \\
      &       &        & 21   & 1000 &  69752 \\ \hline \hline
\multirow{6}[0]{*}{30000} &
\multirow{6}[0]{*}{5871}  &
\multirow{6}[0]{*}{220448}&1.8&   50 & 109950 \\
      &       &        & 3.9  &  100 & 110224 \\
      &       &        & 5.5  &  200 & 110225* \\
      &       &        & 10   &  300 & 110225* \\
      &       &        & 18   &  600 & 110225* \\
      &       &        & 32   & 1000 & 110225* \\ \hline \hline
\multirow{5}[0]{*}{40000} &
\multirow{5}[0]{*}{14259} &
\multirow{5}[0]{*}{299912}&3.1&  50 & 149666 \\
      &       &        & 5.4  &  100 & 149954 \\
      &       &        & 13   &  300 & 149957 \\
      &       &        & 26   &  600 & 149957 \\
      &       &        & 37   & 1000 & 149957 \\
\enddata
%\tablenotetext{a}{Adjusted for inflation}
\tablecomments{* $\beta$-graph files have identical md5sums. NGL returns
  the list of connections for each particle, the pairs (15, 99) and
  (99,15) appear in the catalog. XDL returns only unique connections,
  it means that in the output file has the half of the length, then, for
example, only (15,99) appears in the output file to avoid repetitions.}
\end{deluxetable}


\begin{figure}
  \plotone{extra_connection.pdf}
  \caption{Black crosses shows the new pair of points connected with the
    XDL graph finder. This is the unique difference with the NGL graph
    finder.\label{fig:extra_connection}}
\end{figure}


\begin{figure}
  \plotone{Particle_density_function_log2.pdf}
  \caption{The void density function. It is normalized to the volume
    of the catalog (a sphere of radius $R=100\mathrm{Mpc/h}$).
    \label{fig:void_density_function}}
\end{figure}

\begin{figure}
  \plotone{Particle_density_function_log.pdf}
  \caption{The void density function. It is normalized to the volume
    of the catalog (a sphere of radius $R=100\mathrm{Mpc/h}$).
    \label{fig:void_density_function}}
\end{figure}



\end{document}
