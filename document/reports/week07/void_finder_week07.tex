\documentclass[preprint]{aastex62}
\usepackage[utf8]{inputenc}
\usepackage{lineno}
\usepackage{multirow}
\newcommand{\vdag}{(v)^\dagger}
\newcommand\aastex{AAS\TeX}
\newcommand\latex{La\TeX}
\graphicspath{{./}{figures/}}
\shorttitle{Weekly Activity Report}
\shortauthors{Felipe L. G\'omez-Cort\'es}

\linenumbers
\begin{document}

\title{A $\beta$-Skeleton LSS Void Finder\\Week 07\\Beta-Skeleton Repo. \& Running in the Cluster.}

\correspondingauthor{Felipe Leonardo Gómez-Cortés}
\email{fl.gomez10@uniandes.edu.co}

\author{Felipe Leonardo Gómez-Cortés}
\affiliation{Physics Department, Universidad de Los Andes, Bogotá, 111711, Colombia}

%\author{Jaime E. Forero-Romero}
%\affiliation{Physics Department, Universidad de Los Andes, Bogotá, 111711, Colombia}
%
%\author{Xiao-Dong Li}
%\affiliation{School of Physics and Astronomy, Sun Yat-Sen University, Guangzhou 510297, P. R. China}

\section*{To-Do list}

Tasks for the week:
\begin{itemize}
    \item Create a new repo with the light version of Xiao-Dong Li's LSSCode.
    \item Calculate 1-Skeleton over 125 spherical catalogs (from Abacus Plank Simulation).
    \item Find volume and triaxial parameters of present voids.
    \item Find and Compare Void abundance (by volume) with references.
    \item Take a rest.
\end{itemize}

\section{The HackingLSSCode repository}

I have checked the Xiao-Dong Li's repository LSSCodes (hereafter as XDL). It has many features to do
hard statistics over LSS catalogs like two and three point correlation functions,
$\beta$-Skeleton calculator, redshift corrections (using supposed different cosmologies),
just for name few of them.

The code was written in Fortran90, once is compiled has one executable binary file for
each statistical feature. The $\beta$-Skeleton is of particular interest for my thesis
and ongoing research at the Astroandes Group. So I create a light version of the original
code that only calculates the $\beta$-skeleton without the other features.

This version can be compiled using the gnu compiler \textbf{gfortran} instead the privative
compiler \textbf{ifort}. It works on the Uniandes HPC Magnus cluster. The url of the
``lightwight'' version of the code is available at: 
\url{https://github.com/flgomezc/HackingLSSCode}

\section{Calculate 1-Skeleton for 125 shperical Abacus catalogs}

Once having the $\beta$-skeleton calculator installed on the Magnus Cluster, I wrote a
\href{https://github.com/flgomezc/master_thesis/blob/master/15_running_on_cluster/125_spherical_catalogs.py}
{python main program} to download the catalogs from the \href{https://github.com/forero/abacus/}{repo},
create the Random Catalog using the proportion 2:1 (two random particles per halo point).

The whole process (download + calculations) takes $\sim 34$ minutes, almost 1000 times faster
tha using NGL. With NGL it would have taken $\sim 24$ days, 4.5 hours per catalog over 125 catalogs.

\section{Find Volume and Triaxial parameters of existing voids}

Ongoing. An issue happend with the previous algorithm because of the output format of XDL:
the $\beta$-Skeleton index has not repeated connections. The $\beta$-Skeleton NGL index lists
the connection for each point in the dataset. NGL index is a twice larger than XDL index. 

This issue makes the algorithm identify voids made of a single particle. By the algoritm definition
that is not possible. The bug was fixed by extending the XDL $\beta$-Skel index with the permuted
columns.

Once this problem was solved, some huge voids appeared in the ``Catalog 0''. (the same where
previos NGL calculations where performed).

I have been thinking about using a higher $\beta$ value (maybe $\beta=2$?) or returnig to the
ratio 1:1 between random points and observed points.

\section{Find and compare Void Abundance graph with references}

Some references where found. The Void Volume Abundance graph is not ready, I have not finished yet
the volume calculator. About the references, there is not an standar scale among them.

\href{ArXiv 1902.04585}{https://arxiv.org/abs/1902.04585} Abundances from simulaions. $z = 0, 0.5, 1.0 \text{ \& } 1.5$

\href{ArXiv 1812.05532}{https://arxiv.org/abs/1812.05532} Observational Abundances from SDSS. $0.02<z<0.3$

\href{ArXiv 1807.02938}{https://arxiv.org/abs/1807.02938} Abundances from simulations. $z = 0, 1.0, 2.2$ and others.

\href{Arxiv 1710.01730}{https://arxiv.org/abs/1710.01730} Abudances from sumulations. $z=0.0, 0.5$

\section{Take a rest}

Ongoing.


\end{document}
