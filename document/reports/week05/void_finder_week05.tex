\documentclass[preprint]{aastex62}
% \documentclass[manuscript]{aastex62}
%%  twocolumn, manuscript, preprint, preprint, modern and RNAAS
\usepackage[utf8]{inputenc}
\usepackage{lineno}
\newcommand{\vdag}{(v)^\dagger}
\newcommand\aastex{AAS\TeX}
\newcommand\latex{La\TeX}
%% Tells LaTeX to search for image files in the 
%% current directory as well as in the figures/ folder.
\graphicspath{{./}{figures/}}
%% Reintroduced the \received and \accepted commands from AASTeX v5.2
%%\received{January 1, 2018}
%%\revised{January 7, 2018}
%%\accepted{\today}
\shorttitle{Weekly Activity Report}
\shortauthors{Felipe L. G\'omez-Cort\'es}

%\renewcommand{\abstractname}{RESUMEN}
%\renewcommand{\figurename}{Figura}

\linenumbers
\begin{document}

\title{Weekly Activity Report\\Week 4\\Tagging voids using the Random Point Algorithm\\and the 1-Skeleton}

\correspondingauthor{Felipe Leonardo Gómez-Cortés}
\email{fl.gomez10@uniandes.edu.co}

\author{Felipe Leonardo Gómez-Cortés}
\affiliation{Physics Department, Universidad de Los Andes}
\collaboration{Master Student}

\nocollaboration

\author{Jaime E. Forero-Romero}
\affiliation{Physics Department, Universidad de Los Andes}
\collaboration{Advisor}


\begin{abstract}
  A kind of ``Volume Function'' is plotted. Installing libraries on the Uniandes HPC.
  
\end{abstract}


\section{A kind of Volume Function}

The algorithm ran again over the 3N:N dataset, (by mistake it has run over the 2N:N
dataset) (fig \ref{Tagging voids}). A huge void is detected in the 2N:N dataset, this
void is a big patch over the sphere's surface. It was suposed to be related to a
percolation phenomenom in the 1-Skeleton formation, due the higher density,
random-random point conections on the surface, where voids are trunked.

It was proposed to run the algorithm again, but having a slightly diminished radius
for the Random Points Catalog ($R_{\mathrm{rnd}} = 0.95 R_{\mathrm{obs}}$).

For the record, running times of the $\beta$-Skeleton library on a sphericall
distribution, along to the number of detected voids.

A kind of volume function was calculated asuming that the volume of the void is directly
proportional to the number of particles. Then the void volume is compared to the total
catalog volume in order to get a volume fraction having the same scale and allowing comparissons
between the three datasets. (The straightforward interpretation is particle fraction
instead volume fraction). In the figure \ref{fig:volume_function} the volume fraction
function is shown. Is early to make conclusions, due to frontier effects.

The outer voids are connected by the algorithm. A proposed solution is to create a random
point catalog slightly smaller (in volume) than the observed catalog. Those detected
outer voids will be trunked, they must be deprecated on the statistics at that time.



Is not possible to make a good estimation using only three points, but seems interesting
that the higher number of detected voids was achived with the ratio 2:1. Another
intriguing area is the low volume voids regimen. Is there a sweet spot in the ratio that
maximises the number of small voids? Few runnings more are necessary.

\startlongtable
\begin{deluxetable}{c|c c}
\tablecaption{Running Time of NGL $\beta$-Skeleton \label{tab:running_time}}
\tablehead{
\colhead{Number of Particles} & \colhead{Time (s)} & \colhead{Detected Voids}\\
}
%\colnumbers
\startdata
 5000 & 35    & n.a.\\
10000 & 189   & n.a.\\
20000 & 1717  & 142\\
30000 & 5871  & 165\\
40000 & 14259 & 163\\
\enddata
%\tablenotetext{a}{Adjusted for inflation}
\tablecomments{Running on my laptop: Acer Aspire 4750 (Core i5 2nd gen.). The first
two runs where in order to measure running time.}
\end{deluxetable}

\begin{figure}
  \gridline{
    \fig{tagging_voids_1N.pdf}{0.48 \textwidth}{(a)}
    \fig{tagging_voids_2N.pdf}{0.48 \textwidth}{(b)}}
  \gridline{
    \fig{tagging_voids_3N.pdf}{0.48 \textwidth}{(c)}}
  \caption{Tagging Abacus-Cosmos Voids. This is the first test over ``real''
    data, not a toy model. The LSS shows highly dense packed halos, where
    voids have almost none of them. The idea is to increase the ratio of
    Random Points over Catalog points to study (in later weeks) how void
    detection can be affected. (a) Ratio 1:1 detected 142 voids,
    (b) ratio 2:1 detected 165 voids and (c) ratio 3:1 detected 163 voids.
    The catalog is a spherical set of points, with radius $100 \mathrm{Mpc/h}$,
    N halos = 9981, N random points = 10000 (a), 2N (b) and 3N (c).
    \label{Tagging voids}}
\end{figure}

\begin{figure}
\plotone{volume_function_log}
\caption{A kind of Volume Function. The volume fraction is calculated as the
  particle number in the halo over the total Random Points in the catalog.
  The percolaton over the frontier can be seen as four big voids (2:1 in orange)
  (around the 4\% each one), and
  one huge void (3:1 green) that occupies almost the $20\%$ of the total volume.
  \label{fig:volume_function}}
\end{figure}


\section{To do:}

\begin{itemize}
\item Change the shape of the Random Points Catalog (maybe outer voids are
  connected by mistake) taking a sphere of radius $R_{\mathrm{rnd}} = 0.95R_{\mathrm{obs}}$
\item Run using other number of random particles.
\item Think about how to calculate volume (not just particle number times density).
\item Calculate Intertia Tensor and find eigen-values.
\end{itemize} 

%\section{Conclusions}



 % \nocite{*}

  %\begin{thebibliography}{20}

    %\bibitem[Bos et al.(2012)]{Bos2012} Bos, P. et al. \ 2012, \mnras, 426, 440 % Testing cosmologies using void ellipticity

  %\end{thebibliography}                                                           
                       

%\listofchanges

\end{document}
