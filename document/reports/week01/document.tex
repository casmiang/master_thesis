\documentclass[preprint]{aastex62}
% \documentclass[manuscript]{aastex62}
%%  twocolumn, manuscript, preprint, preprint, modern and RNAAS
\usepackage[utf8]{inputenc}
\newcommand{\vdag}{(v)^\dagger}
\newcommand\aastex{AAS\TeX}
\newcommand\latex{La\TeX}
%% Tells LaTeX to search for image files in the 
%% current directory as well as in the figures/ folder.
\graphicspath{{./}{figures/}}
%% Reintroduced the \received and \accepted commands from AASTeX v5.2
%%\received{January 1, 2018}
%%\revised{January 7, 2018}
%%\accepted{\today}
\shorttitle{Weekly Activity Report}
\shortauthors{Felipe L. G\'omez-Cort\'es}

%\renewcommand{\abstractname}{RESUMEN}
%\renewcommand{\figurename}{Figura}


\begin{document}

\title{Weekly Activity Report\\A Large Scale Structure Void Identifier for Galaxy Surveys
  Based on the $\beta$-Skeleton Graph Method}

\correspondingauthor{Felipe Leonardo Gómez-Cortés}
\email{fl.gomez10@uniandes.edu.co}

\author{Felipe Leonardo Gómez-Cortés}
\affiliation{Physics Department, Universidad de Los Andes}
\collaboration{Master Student}

\nocollaboration

\author{Jaime E. Forero-Romero}
\affiliation{Physics Department, Universidad de Los Andes}
\collaboration{Advisor}


\begin{abstract}
  The beta parameter graph, its definition.

  The testing catalogs.

  How the graph changes while variying $\beta$.

  
  
  \keywords{ Beta Skeleton Graph, Beta Parameter}

\end{abstract}

\newpage

\section{The Algorythm}

The algorytm uses the NGL library % #################################################(CITAR NGRAPH.ORG)
to create the $\beta$-Skeleton graph of a given set of points.
Then, the histogram of connection lenghts is created
(in a similar fashion to the two-point correlation function)
for hhe 0.90-Skeleton and 1-Skeleton graphs.
The average lenght for points inside a structure is defined
(using the 1-Skeleton), and then those small conections are
removed from the 0.9-Skeleton graph. The remainig graph
contains the points over the surface of the voids.



\subsection{Mock Catalog}
In order to test the algorythm, two catalogs of points where
created to emulate voids in the LSS.

The emulated space is a cubic region of 60 Mpc/h.
There where placed $\sim 5 \times10^4$ points to have a similar
volume density of points to the halo volumetric density in
the AbacusCosmos simulations. ($8.7\times10^6$ halos in a cubic
box of 720 Mpc/h length,
$2.335\times10^{-2}\mathrm{halo/(Mpc/h)^3}$).
Points where placed using an uniform density probability distribution.

The first set of points has an spherical empty region of radius
20 Mpc/h centered in the middle of the volume.

The second catalog has four non-overlapping spherical empty
regions of radius between 8 and 20 Mpc/h.


\subsection{The $\beta$-Skeleton Graph}

The $\beta$-Skeleton graph is defined by the relative distance
between points and a geometrical criterion using a real
parameter $\beta\geq 0$.

Two points are connected in the graph if there is an empty region
between them, without any other point. The shape of the empty region
is function of the $\beta$/parameter.

The authors define the $\beta$-skeleton as folows:

``The so-called lune-based $\beta$-skeleton is a one-parameter
generalization of the RNG [Relative Neighborhood Graph] and
GG [Gabriel Graph], defined as follows:

\begin{itemize}
\item For $0<beta<1$, the empty region is the intersection of all
  d-balls with diameter $d(p,q) / \beta$ that have p and q on the
  boundary.
\item For $\beta \geq 1$, the empty region is the intersection of
  two d-balls with diameter $\beta d(p,q)$ centered at
  $(1-\frac{\beta}{2})p + \frac{\beta}{2}q$ and
  $\frac{\beta}{2}p + (1-\frac{\beta}{2})q$.
\end{itemize}.

It follows that $\beta = 2$ gives the RGN, while $\beta = 1$ is
the GG.''
  
In the limit when $\beta$ tends to zero, every point is conected
to each point on the set, it corresponds to the graph used in
the classic two-point correlation function. Each point has N
connections. (With N as the number of points in the set).

When $/beta$ is increasing, the number of connections per point
is reduced. The first connections to vanish are the longer ones,
while the near connections persists.

\subsection{Structure deppendence of $\beta$ parameter}

\begin{figure}
  \gridline{
    \fig{beta_0_70.png}{0.3 \textwidth}{(a)}
    \fig{beta_0_90.png}{0.3 \textwidth}{(b)}
    \fig{beta_0_99.png}{0.3 \textwidth}{(c)}}
  \gridline{
    \fig{beta_1_00.png}{0.3 \textwidth}{(d)}}
  \gridline{
    \fig{beta_3_00.png}{0.3 \textwidth}{(e)}
    \fig{beta_6_00.png}{0.3 \textwidth}{(f)}
    \fig{beta_9_00.png}{0.3 \textwidth}{(g)}}
  \caption{Beta variations}
  % ###################################### Figures must be in PDF format.
  % ###################################### And thight margins.
\end{figure}



  \nocite{*}

  \begin{thebibliography}{20}

    %\bibitem[Bos et al.(2012)]{Bos2012} Bos, P. et al. \ 2012, \mnras, 426, 440 % Testing cosmologies using void ellipticity

  \end{thebibliography}                                                           
                       

%\listofchanges

\end{document}
