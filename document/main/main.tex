\documentclass[preprint]{aastex62}
\usepackage[utf8]{inputenc}
\usepackage{lineno}
\usepackage{multirow}
\newcommand{\vdag}{(v)^\dagger}
\newcommand\aastex{AAS\TeX}
\newcommand\latex{La\TeX}
\graphicspath{{./}{figures/}}
\shorttitle{A LSS Void Identifier based on the $\beta$-Skeleton Graph Method}
\shortauthors{G\'omez-Cort\'es, Forero-Romero, Li}

\linenumbers
\begin{document}

\title{A Large Scale Structure Void Identifier for Galaxy Surveys Based on the 
$\beta$-Skeleton Graph Method}

\correspondingauthor{Felipe Leonardo Gómez-Cortés}
\email{fl.gomez10@uniandes.edu.co}

\author{Felipe Leonardo Gómez-Cortés}
\affiliation{Physics Department, Universidad de Los Andes}

\author{Jaime E. Forero-Romero}
\affiliation{Physics Department, Universidad de Los Andes}

\author{Xiao-Dong Li}
\affiliation{Shcool of Physics, Korean Institute for Advanced Study}


\begin{abstract}

\end{abstract}


\section{Introduction}


\section{The Algorithm}

This void identifier is based on the $\beta$-Skeleton graph. The input is a
set of 3-dimensional points (galaxy/halo catalog) in the coordinate space with
proper distances. This is the Observed Catalog (\textbf{OC}). Once is defined
the volume and shape of the \textbf{OC}, this space is populated with random
points following a uniform probability density function, this is the Random
Catalog (\textbf{RC}). The Full Catalog (\textbf{FC}) is made by assembling
the OC and RC. The $\beta$-Skeleton graph is built over the FC without any
discrimination connecting observed and random points, the graph is stored as
\textbf{fcBSkel}. From the fcBSkel, all the observed points and their
connections are deprecated; it means that some neighbour random points are
rejected. The remaining random points are conected only with other random
points, they conform the voids.

This method has two parameters: $\beta$ and the ratio between random
points and observed points.


\subsection{Testing methods with the Void Toy Models}

The Li's $\beta$-Skeleton library \citep{Li..Github..2014} is written in
Fortran 90. The calculation of the 1-Skeleton graph with $\sim 10^4$ points
in a regular laptop Linux machine (Core i5, 2nd gen.) takes 192 seconds.
Input files of three dimensionless columns ($x$, $y$, $z$): $\mathbb{R}^3$
points.

Observational Toy Model Catalogs where created to test if the algorithm was
viable. 

The first toy model catalog was generated with a spherical void into inside
a cubic box (100 arbitrary units of lenght) filled with $N_{obs}$ random
uniformly distributed points.

\subsubsection{The weak method}
A first attempt was an algorithm based on
the comparison of two $\beta$-Skeleton graphs ($\beta_1 < 1$), $(\beta_2 >= 1$),
(fig. \ref{fig:first_algorithm}). This method was able to detect multiple
spherical voids, but was unable to detect the whole enclosing surface of
elliptical voids, the second testing toy model. 

\begin{figure}
  \plottwo{first_algorithm/beta_0_9.png}
          {first_algorithm/beta_1_0.png}
  \caption{Weak method: Long connections dissapear in the
    $\beta$-Skeleton graph as $\beta$ increases.
    Left: Connections with $\beta=0.9$. Right: Connections with $\beta=1.0$.
    The low value of $\beta$ gives the graph long connections across the
    spherical void. They dissapear with $\beta = 1.0$, known as the Gabriel
    Graph. \label{fig:first_algorithm}} 
\end{figure}

\subsubsection{The robust method}
The second and definitive algorithm relies on random points filling voids.
The algorithm sucessfully recognized the spherical void, and also identified
some other underdense regions, figure \ref{fig:First void recognition}-left.
A limit resolution arises, related to the number of random points, it was
shown finding small spherical voids placed in different locations, figure
\ref{fig:First void recognition}-right.

\begin{figure}
  \plottwo{single_sphere_N5000_R37_BL100/RandomPoints_Recognized_Void_1_0-skeletons.pdf}
          {multiple_spheres_limit_radius_N5000_R5_10_15_20_BL100/RandomPoints_Recognized_Voids.pdf}
  \caption{The robust method identifies voids by populating the volume of the
    observed catalog with random points. Parameters: $\beta=1$ and $n_{rnd}=1$.
    Left: Random Points identified as ``Void Points''. They are points
    from the Random Catalog
    connected by the 1-Skeleton only to other Random Catalog points. 
    N=5000 points, BoxLength = 100 units and void radius R=37 units.
    Right: Points identified as ``Void Points''. This tests aims to check
    the resolution of the algorithm. Some scattered micro-voids appear.
    The two big voids are recognized (R=15,20), but small structures doesn't
    appear (R=5,10).
    N=5000, BoxLength = 100, voids radius R= [5, 10, 15, 20].
    \label{fig:First void recognition}}
\end{figure}

The robust method cand find irregular voids.
The last void toy model used overlaping ellipsoidal voids with semiaxes
ratios 1:0.7:0.5, different orientations and major semiaxes of 40, 30,
30 and 20 arbitrary units, generating a big irregular void. The OC and a
slice is shown at figure \ref{fig:scatter_OC}.

\begin{figure}
  \plottwo{explaining/toy_model_OC-3d.pdf}{explaining/toy_model_OC.pdf}
  \caption{A Irregular Void in the testing toy model. Observed points
    (fake galaxies) in blue. Left: 3D scatter-plot
    of the full sample of OC.
    Right: A slice of the OC at $z=50\pm5$ arbitrary units.
    \label{fig:scatter_OC}}
\end{figure}

The Random Catalog was created using the same number of random points as
the observed points.$N_{rnd}=N_{obs}$ (Fig.\ref{fig:scatter_OC+RC}).It follows
to the recognition of the main structures (ellipsoidal voids overlapping as
a bigger one) plus small unexpected structures. Those are clue of the
relevance of the ratio $n_{rnd} = N_{rnd}/N_{obs}$ as free parameter of the
algorithm. Having $n_{rnd}>1$ means a higher resolution in the shape of the
identified voids, but also smaller voids are detected in the catalog; are
they artifacts or true voids? The answer deppends on the void definition itself.
The initial test is performed with $n_{rnd}=1$.

\begin{figure}
  \plottwo{explaining/toy_model_OC+RC-3d.pdf}
          {explaining/toy_model_OC+RC-slice.pdf}
          \caption{The Random Catalog (red) of points populates the whole volume
            of the (blue) OC. In this case, we have the same number of Random
            Points and Observed Points. Left: 3D scatter-plot of the OC plus
            RC. Right: A slice of the OC plus RC at $z=50\pm5$ arbitrary units.
            \label{fig:scatter_OC+RC}}
\end{figure}

\begin{figure}
  \plottwo{explaining/toy_model_fcBSkel-slice.pdf}
          {explaining/toy_model_truevoidsBSkel-slice.pdf}
          \caption{A slice ($z=50\pm5$ arbitrary units) of the
            $\beta$-Skeleton of the full catalog (left).
            After removing the observed points and their connections, the
            remaining points are the void points (right).
            \label{fig:beta_skel_slices}}
\end{figure}

\subsection{Tagging Voids}

So far, the method can isolate the true void points: RC points that are not
connected with the OC according to the $\beta$-Skeleton graph. The
identification of structures uses the same graph, using a recursive search of
connected neighbours; a recursive search of friends of friends.

\begin{figure}
  \plottwo{explaining/recursion2_all_identified_voids.pdf}
          {explaining/recursion2_big_identified_voids.pdf}
  \caption{Irregular voids Voids found in the OC. This algorithm can indentify
           irregular structures; a giant leap ahead the previous algorithm.
           The irregular voids are made of overlapping ellipsoidal voids, with
           semiaxes proportion 1:0.7:0.5 as the literature suggets.  
           N=5000, BoxLength = 100, the voids major semiaxes where chosen as
           R= 40, 30, 30 and 40 arbitrary units.
           Left: A total of 41 voids are identified, all of them with at least
           one random point. To take into account, only four toy ellipsoidal
           voids where placed in the OC, two of them overlapping.
           Right: Voids with 10 or more random points.
    \label{fig:}}
\end{figure}

%We did not achieve the stage of exploring voids with sparse galaxies in the
%previous algorithm, but now, a void with sparse galaxies is easily identified
%as a ``plum pudding'', where the void is the pudding with frontier points
%enclosing the ``plums'' (sparse galaxies), 


%###################################  Friday, 2019-08-16 22:15

\section{Finding Voids in Simulations}

\subsection{Abacus-Cosmos Simulation}

From a snapshot of the Abacus-Cosmos simulation with fiducial cosmology, a set of
halos is selected, within a sphere of radius $R=100\mathrm{Mpc/h}$ and a $M_{cut}$
given to get $\sim 10^4$ halos.

The Random Catalog is generated, taking care of having a uniform space
distribution. A common error is to generate the radius using a uniform
distribution from 0 to 1 and scaling by a factor $R$,
but must be taken instead the cubic root of the randon number between 0 and 1,
then multiplied by $R$. Another error is generating the angles again, taking a
random number between zero and one, then scaling by $2\pi$. This error can be
avoided by generating a randon number between 0 and 1 as the cosine of $\theta$,
then calculating the angle as the arc-cosine. The unitary vector can be
calculated using a gaussian distribution for x, y and z (centered at zero
with the same deviation for them three).  % ################################## <<--- Search for citation here.
The norm is calculated as usually, by taking the square root of the sum of
the coordinates, then dividing each coordinate by the norm of the vector.

The algorithm ran over the ``real'' dataset using the rations 1:1, 2:1
and 3:1 for the number of random points vs. catalog points.

\subsection{Running over a single spherical cut}

\begin{figure}
  \gridline{
    \fig{abacus/tagging_voids_1N_N.pdf}{0.32 \textwidth}{(a)}
    \fig{abacus/tagging_voids_2N_N.pdf}{0.32 \textwidth}{(b)}
    \fig{abacus/tagging_voids_3N_N.pdf}{0.32 \textwidth}{(c)}}
  \caption{Tagging Abacus-Cosmos Voids. This is the first test over ``real''
    data, not a toy model. The LSS shows highly dense packed halos, where
    voids have almost none of them. The idea is to increase the ratio of
    Random Points over Catalog points to study (in later weeks) how void
    detection can be affected. (a) Ratio 1:1, (b) ratio 2:1 and (c) ratio 3:1.
    The catalog is a spherical set of points, with radius $100 \mathrm{Mpc/h}$,
    N halos = 9981, N random points = 10000 (a), 2N (b) and 3N (c).
    \label{fig:Tagging Abacus Cosmos Voids}}
\end{figure}

A kind of percolation phenomenon is apreciated at figure
\ref{fig:Tagging Abacus Cosmos Voids}. Having a similar number of Random Points
and Observational Points (a) conducts to sucessfull void identification, while
a higher ration 2:1 (b) and 3:1 (c) 

\subsection{Running over 64 spherical cuts}

\subsection{Running over Corner Plots}







\section{Finding Voids in Galaxy Redshift Surveys}


\subsection{SDSS Planck 2015 data}

The SDSS has $\sim 57000$ galaxies in the truncated cone section from RA 0 to $50\deg$,
and DEC from $-40\deg$ to $+40\deg$. Using Planck-2015 cosmology, reshift was converted into the
range of distance from 0 to 300 Mpc. (figure \ref{fig:Planck15_cat}).
\begin{figure}
  \plotone{figures/Planck15_catalog.png}
  \caption{Dataset from SDSS. Comoving distances are in Mpc, calculated from the redshift
    measurements using the module ``cosmology'' from the library ``astrpy'' with Planck15
    cosmological parameters.
    \label{fig:Planck15_cat}}
\end{figure}

The void size density function is calculated as the probability to find a void
of a given radius in to the total volume. There is a discrepancy between our density function
and Platen (2008) results, maybe they did not normalize by the volume.
Figure \ref{fig:size_density_function}.

\begin{figure}
  \plottwo{figures/void_size_density_function.pdf}{figures/platen_void_size_density_function.png}
  \caption{Void size density function. Left, our analysis. Right, Platen et al (2008).
    \label{fig:size_density_function}}
\end{figure}

The void ellipticity is calculated as $\epsilon = 1 - c/a$, with $a$ the biggest semi-axe, and
$c$ the smallest semi-axe. They two graphs seem to have similar values. The probability density
is not normalized by the volume. Figure \ref{fig:ellipticity}.

\begin{figure}
  \plottwo{figures/void_ellipticity.pdf}{figures/platen_void_ellipticity.png}
  \caption{Void Ellipticity. Left, analysis of voids found with our algorithm.
    Rigth, Platen et al (2008).
    \label{fig:ellipticity}}
\end{figure}

The last comparison was made using the two axis ratios to find how prolate, oblate or close
to spheroid. Our results are pretty close to Platen results (2008).

\begin{figure}
  \plottwo{figures/void_two_axis_ratios.pdf}{figures/platen_void_two_axis_ratios.png}
  \caption{Scatter diagram of two axis ratios. In the approximation of the voids
    as ellipsoidal forms, the semi-axes are labeled, from shorter to larger, as
    $a$, $b$ and $c$. A perfect sphericall void should be placed at the rigth top
    corner of the diagram. 
    \label{fig:two_axis_ratios}}
\end{figure}



\software{LSSCode \citep{Li..Github..2014}}


\begin{thebibliography}{}

\bibitem[LSSCode (2014)]{Li..Github..2014} Li, Xiao-Dong \ 2014, LSSCode \ \url{https://github.com/xiaodongli1986/LSS_Code}
\bibitem[Astropy Collaboration et al.(2013)]{2013A&A...558A..33A} Astropy Collaboration, Robitaille, T.~P., Tollerud, E.~J., et al.\ 2013, \aap, 558, A33 

\end{thebibliography}


\end{document}
