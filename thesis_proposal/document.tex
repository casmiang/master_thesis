\documentclass[preprint]{aastex62}
% \documentclass[manuscript]{aastex62}
%%  twocolumn, manuscript, preprint, preprint, modern and RNAAS
\usepackage[utf8]{inputenc}
%\usepackage{siunitx}
%\usepackage[spanish]{babel}
%
\newcommand{\vdag}{(v)^\dagger}
\newcommand\aastex{AAS\TeX}
\newcommand\latex{La\TeX}
%% Tells LaTeX to search for image files in the 
%% current directory as well as in the figures/ folder.
\graphicspath{{./}{figures/}}
%% Reintroduced the \received and \accepted commands from AASTeX v5.2
%%\received{January 1, 2018}
%%\revised{January 7, 2018}
%%\accepted{\today}
\shorttitle{A LSS Void Identifier based on $\beta$-Skeleton}
\shortauthors{F. L. Gómez-Cortés}
  
\begin{document}

\title{A Large Scale Structure Void Identifier for Galaxy Surveys
  Based on the $\beta$-Skeleton Graph Method}

\correspondingauthor{Felipe Leonardo Gómez-Cortés}
\email{fl.gomez10@uniandes.edu.co}

\author{Felipe Leonardo Gómez-Cortés}
\affiliation{Physics Department, Universidad de Los Andes}
\collaboration{Master Student}
\collaboration{Code 201324084}

\nocollaboration


\author{Jaime E. Forero-Romero}
\affiliation{Physics Department, Universidad de Los Andes}
\collaboration{Advisor}


%%\begin{abstract}
\section*{Abstract}

  We are living the golden age of observational cosmology. 
  There is a consolidated standard cosmological model ($\Lambda$CDM) that explains the observed
  Large Scale Structure (LSS) of galaxies by introducing dark matter and
  dark energy as the dominant Universe components along with baryonic matter.
  Furthermore, we are able to do precise observatioanl measurements of the 
  cosmological parameters in that model. 
  Most of this success is due to computational cosmology that is now 
  an stablished tool to probe theoretical models and compare them with observations.
  The main features of the LSS can been reproduced in large cosmological N-body simulations.
  
  One of the most striking features in the LSS are voids; irregular 
  volumes on the order of tens of Mpc scales, where the matter density is below the Universe
  average density. 
  Statistics about voids such as its volume, shape and orientation also encode cosmological information.
  For this reason there is a great interest in algorithms that find and characterize voids
  both in simulations and observations.

  The objetive of this work is to develop a new void finder based
  on the $\beta$-Skeleton method.
  The $\beta$-Skeleton has been widely used on image processing,
  recognition and machine learnig applications, it has been introduced
  recently in LSS analysis. It is a fast tool identifiying LSS filamentary structure,
  and promises to be a robust tool to make cosmological tests.
  After developing the void finder we will characterize the $\beta$-skeleton voids
  in simulations and observations. We will also make prediction for the upcoming
  Dark Energy Spectroscopic Instrument about the void population that could be detected
  with that experiment.
  

  \medskip

  Keywords: Large Scale Structure, cosmology, voids, computational astrophysics

  \keywords{ Large Scale Structure, cosmology, voids, computational astrophysics}
  
%%\end{abstract}

  \section*{Resumen}

  Estamos viviendo en la era dorada de la cosmolog\'ia observacional.
  Existe un modelo est\'andar comol\'ogico ($\Lambda$-CDM) consolidado que explica
  las observaciones de la Estructura de Gran Escala (LSS) de galaxias mediante
  la introducci\'on de materia oscura y energ\'ia oscura como las componentes
  dominantes del Universo junto con la materia bari\'onica. M\'as a\'un, somos capaces de
  realizar mediciones precisas de los par\'ametros cosmol\'ogicos de este modelo a partir de
  observaciones. Gran parte de estos alcances es debido a la cosmolog\'ia computacional
  que es ahora una herramienta fuertemente establecida para probar modelos te\'oricos y
  compararlos con las observaciones. Las caracter\'isticas principales de la LSS pueden
  ser reproducidas en grandes simulaciones cosmol\'ogicas de N-cuerpos.

  Una de las caracter\'isticas m\'as prominentes en la LSS son los vac\'ios: vol\'umenes
  irregulares de escalas del orden de decenas de Mpc, donde la densidad de materia est\'a
  por debajo de la densidad media en el Universo. El an\'alisis estad\'istico de propiedades
  de los vac\'ios, como su volumen, forma y orientaci\'on tambi\'en nos puede dar informaci\'on
  cosmol\'ogica. Por esta raz\'on existe un gran inter\'es en algoritmos que encuentren y
  caractericen vac\'ios tanto en simulaciones como en observaciones.

  El objetivo de este trabajo es desarrollar un nuevo buscador de vac\'ios basado en el
  m\'etodo $\beta$-Skeleton. El m\'etodo $\beta$-Skeleton ha sido ampliamente utilizado
  en reconocimiento, procesamiento de im\'agenes y aplicaciones de \textit{machine learning},
  recientemente ha sido introducido en el an\'alisis de LSS. Esta es una herramienta r\'apida
  para identificar estructuras filamentarias en la LSS, y promete ser una herramienta robusta
  para realizar an\'alisis cosmol\'ogicos. Luego de desarrollar el buscador de vac\'ios
  caracterizaremos los vac\'ios del $\beta$-Skeleton en simulaciones y observaciones. Tambi\'en
  realizaremos predicciones para el experimento en desarrollo Dark Energy Spectroscopic
  Instrument (DESI) acerca de la poblaci\'on que podr\'a detectar.

  \medskip

  Palabras clave: estructura de gran escala, cosmología, vacíos,
  astrofísica computacional.


  \section{Introducción}

  Hace poco más de un siglo se concebía el Universo estacionario tan grande como nuestra
  Vía Láctea, la cosmología había sido construída sobre especulaciones y apenas empezaban
  las primeras observaciones con métodos y técnicas lo suficientemente sistemáticas y
  rigurosas para apoyar modelos cosmológicos.
  Las contribuciones teóricas de Albert Einstein y las observacionales de Edwin 
  Hubble cambiaron por completo el paradigma del Universo en 1929. El Universo era ahora un ente
  dinámico en constante cambio, que tuvo un origen en algún momento. Lemaître propuso
  en 1931  un modelo del Universo basado en los modelos de Friedman, donde el universo tendría
  su origen en un ``átomo primigenio'', una singularidad donde toda la energía y materia
  del Universo estaría concentrada, a una temperatura y una densidad muy altas. Afuera
  de esta densidad no existía el espacio ni el tiempo. Esta singularidad se expande,
  a medida que esto sucede la densidad de energía y materia se reducen.
  En intentos de desprestigiar este modelo, Hoyle lo llamó  ``Big Bang'',
  neologismo que ganó popularidad y ha perdurado hasta la fecha sin el lastre peyorativo.

  En 1933, Zwicky medía la velocidad de dispersión de galaxias en el clúster Coma
  para calcular la masa total por el teorema del virial. Luego comparó este resultado
  con la masa estimada a partir de la relación Masa-Luminosidad, pero el resultado de
  los dos cálculos discrepaba. En el clúster había mucha más materia que la que daba cuenta
  de la materia luminosa ordinaria  asociada a las galaxias. Esta fue la primera evidencia de la
  existencia de la materia oscura.

  En 1946, George Gamow desarrolló un modelo para la nucleosíntesis basado en el modelo
  del Universo de Lemaître, con la mecánica cuántica desarrollada hasta la fecha se
  podía predecir temperaturas y densidades altas donde un gas de partículas elementales
  (protones, neutrones y electrones) podría dar origen a los núcleos de los elementos
  más abundantes del Universo: hidrógeno y helio. Chandrasekar y Henrick habían sentado las
  basess para el desarrollo de la nucleosíntesis unos años atrás.  En 1948, Alpher y Herman
  encontraron que en cierto punto la interacción entre la materia y la radiación dejarían
  un remanente. 
  
  En 1965 se detectó por primera vez ese remanente, la radiación cósmica de fondo (CMB).


  En 1960, von Hoerner realizó las primeras simulaciones en computador resolviendo problemas
  de N-cuerpos (con 16 cuerpos inicialmente) ligados gravitacionalmente. Poco tiempo después
  se empezaron a resolver problemas de clusters de galaxias y cúmulos globulares de estrellas
  con los trabajos individuales de Aarseth, Wilem, Lecar y Almada entre otros pioneros. A
  mediados de los 80's se realizaban simulaciones con $N = 5.000$ cuerpos.
  \citep{Aarseth2003}.

  
  
  En los 80's se lanzó la misión COBE para estudiar con mayor detalle esta radiación,
  y a la par empezaron a realizarse grandes proyectos para mapear el cielo como el
  \textit{Center for Astrophysics (CfA) Survey} aprovechando la reciente tecnología de
  cámaras CCD. Se encontraron grandes estructuras de galaxias ligadas gravitacionalmente,
  del orden de $\sim 100 h^{-1}\rm Mpc$.
  Posteriormente, en los 90's, en el mapeo de \textit{Las Campanas Redshift Survey (LCRS)} empezó a
  definirse la noción de Estructura del Universo a Gran Escala (\textit{Large-Scale
    Structure LSS}); las galaxias no están puestas aleatoriamente en el espacio, obedecen
  a una estructura de red cósmica con filamentos y paredes donde se observa una gran
  densidad de galaxias y vacíos en la red donde hay pocas galaxias. \citep{Schneider2014}


    


  WMAP, Plank
  Millenium, Bolshoi-Plank
  
  

  \section{Estado del Arte}

  \section{Marco Teórico}

  Actualmente se dispone tanto de catálogos de halos de materia oscura de simulaciones como de catálogos de galaxias de observaciones.

  Se dispone de la librería ``NGL''\citep{ngl} (Neighborhood Graph Library)
  para calcular la estructura $\beta$-Skeleton. Esta librería es de uso libre, está escrita en C++.
  Inicialmente fue desarrollada para estudiar topología de conjuntos de datos con un número de muestra
  pequeño. Puede encontrar vecinos en los conjuntos de puntos usando distintos métodos como los
  Grafos de Gabriel y el $\beta$-Skeleton, estos métodos resultan computacionalmente más económicos
  y rápidos que otros métodos más robustos como la trianguación de Delaunay.

  El método $\beta$-Skeleton ha sido introducido en astrofísica recientemente \citep{Fang2018} para
  encontrar el grafo de la LSS en catálogos de halos de materia oscura de distintas simulaciones.
  Ha probado ser una herramienta útil para detectar estructuras subyacentes diferentes en
  simulaciones cosmológicas generadas por métodos distintos.
  

  \section{Objetivos}

  \subsection{Objetivo Principal}
  Desarrollar un nuevo buscador de vacíos de la red cósmica (LSS) en catálogos de galaxias
  basado en el método $\beta$-Skeleton.
  
  \subsection{Objetivos Específicos}

  \begin{itemize}

      \item Identificar y catalogar vacíos de la red cósmica en catálogos de halos de materia oscura de
  simulaciones y en catálogos de galaxias desde observaciones.

      \item Calcular el parámetro cosmológico asociado a la energía oscura a partir del catálogo de
  vacíos de la red cósmica

    \item  Estimar los posibles resultados observacionales que pueda medir el experimento DESI.
  \end{itemize}
  
  \section{Metodología}

  La primera parte de este trabajo consiste en desarrollar el código del buscador de vacíos
  en la red cósmica. Aquí se incluye una etapa de calibración de parámetros libres del buscador.
  En la segunda parte se utilizará el código para obtener catálogos
  de vacíos de la red a partir de catálogos de halos de materia oscura de simulaciones o
  catálogos de galaxias de observaciones, mediante análisis estadístico de los catálogos de
  vacíos se compararán simulaciones y observaciones.
  En la tercera parte se hará un análisis estadístico para determinar parámetros cosmológicos.
  Finalmente se podrá dar un estimado de las observaciones que podrá llegar a medir el
  experimento DESI.

  \subsection{Desarrollo del código}

  El código será escrito en Python3 o C++, podrá ejecutarse en el Colaboratorio o en el HPC de la
  facultad de ciencas de la Universidad de los Andes.
  
  Para desarrollar el código se parte de la lectura de archivos estándar de catálogos de halos
  de materia oscura o distribución de galaxias. Estos puntos se ubican en un espacio tridimensional.

  Por definición los vacíos cósmicos son regiones con baja densidad de materia, así que se
  estudiarán regiones de un campo escalar. Para esto es conveniente dividir el espacio en
  celdas discretas. El tamaño de las celdas será el primer parámetro a calibrar en el código.
  
  Se aplica el método $\beta$-Skeleton para conectar los puntos en el espacio y trazar los
  filamentos de la red cósmica. En este punto se utilizará la librería NGL \citep{ngl}.
  El parámetro $\beta$ de este método es otro parámetro a calibrar en el código. Se puede
  utilizar como guía el resultado obtenido por \citet{Fang2018}.
  Se transforman estos puntos y filamentos en un campo de densidad de materia. Este primer
  campo escalar de materia es bastante discreto y discontínuo. Será suavizado mediante un
  kernel Gaussiano, el número de veces que sea suavizado el campo de densidad y el tamaño de
  las celdas del kernel serán parámetros a calibrar en el código.

  Una vez suavizado el campo de densidad de materia se identifican los centroides
  de los vacíos de la red. Esto se puede realizar revisando, por ejemplo, puntos intermedios
  en conexiones largas que pueden aparecer en un $\beta$-skeleton con parámetro $\beta$ bajo,
  pero no aparecen en la red obtenida con el valor de $\beta$ apropiado.

  Ya identificados los centroides de los vacíos, se utiliza el método \textit{watershed}
  para identificar las regiones de baja densidad \citep{Sutter2015}. Con este método
  se planta una semilla en los centroides, todas las semillas crecen a la misma velocidad; una
  celda vecina a la vez. Se establecen ciertas reglas, por ejemplo las regiones no crecerán en
  las regiones donde el campo escalar de densidad tenga un valor alto, o se detendrán cuando
  se encuentren con otra región en crecimiento. Otro método a probar para identificar regiones
  es la construcción de esferas dentro de los vacíos de la red. 

  \subsection{Análisis Estadístico de Catálogos de Vacíos de la Red Cósmica}

  Se calculan propiedades de las regiones como tamaño, forma, orientación. (Tensor de momento de inercia) \citep{El-Ad1997}

  Se estudian propiedades de distorsión de esferas hacia elipses por el efecto \citet{AlcockPaczynski1979}.
  \subsection{Análisis de Parámetros Cosmológicos}

  \subsection{Posibles Resultados del Experimento DESI}
  


  

  \section{Cronograma}

  \begin{table}[htb]
    \begin{tabular}{|c|cccccccccccccccc| }
      \hline
      Tareas $\backslash$ Semanas & 1 & 2 & 3 & 4 & 5 & 6 & 7 & 8 & 9 & 10 & 11 & 12 & 13 & 14 & 15 & 16  \\
      \hline
      1 & X & X & X & X & X & X &   &   &   &   &   &   &   &   &   &   \\
      2 &   &   &   &   & X & X &   &   &   &   &   &   &   &   &   &   \\
      3 &   &   &   &   &   &   & X & X &   &   &   &   &   &   &   &   \\
      4 &   &   &   &   &   &   & X & X & X &   &   &   &   &   &   &   \\
      5 &   &   &   &   &   &   &   &   & X & X &   &   &   &   &   &   \\
      6 &   &   &   &   &   &   &   &   &   &   & X & X & X &   &   &   \\
      7 &   &   &   &   &   &   &   &   &   &   &   &   &   & X & X & X \\
      8 & X & X & X & X & X & X & X &   &   &   &   &   &   &   &   &   \\
      9 &   &   &   &   &   &   & X & X & X &   &   &   &   &   &   &   \\
      10&   &   &   &   &   &   &   &   &   & X & X & X & X &   &   &   \\
      11&   &   &   &   &   &   &   &   &   &   &   &   &   & X & X & X \\
      12&   &   &   &   &   &   &   &   &   &   &   &   &   &   &   & X \\
      \hline
    \end{tabular}
  \end{table}

  
  \begin{enumerate}
  \item Desarrollo del código del Buscador de Vacíos Basado en $\beta$-Skeleton.
  \item Calibración de los parámetros del código y comparación con otros buscadores.
  \item Obtención de catálogos de vacíos de la red a partir de simulaciones.
  \item Obtención de catálogos de vacíos de la red a partir de observaciones.
  \item Comparación por análisis estadístico entre simulaciones y observaciones.
  \item Cálculo de la constante cosmológica a partir de catálogo de vacíos de la red en
    observaciones.
  \item Estimación de resultados para el experimento DESI.
  \item Escritura del Documento: Introducción y marco teórico.
  \item Escritura del Documento: Desarrollo y calibración del Código.
  \item Escritura del Documento: Análisis de catálogos de vacíos de la red cósmica. 
  \item Escritura del Documento: Estimación de posibles resultados del experimento DESI.
  \item Escritura del Documento: Conclusiones
  \end{enumerate}
  
  
  \section{Resultados Esperados}

  
  
  \nocite{*}


  \begin{thebibliography}{}

    \bibitem[Aarseth(2003)]{Aarseth2003} Aarseth, S. J. \ 2003, ``Gravitational N-Body Simulations'', Cambridge University Press.
    \bibitem[Alcock-Paczynski(1979)]{AlcockPaczynski1979} Alcock, C. \& Paczy\'nski, B.\ 1979, \nat, 281, 358    
    \bibitem[Correa \& Lindstrom(2012)]{ngl} Correa, Carlos \& Lindstrom, Peter.\ 2011,  IEEE TVCG\ 17,12 (Dec 2011), 1852-1861
    \bibitem[El-Ad \& Piran(1997)]{El-Ad1997} El-Ad, H. \& Piran, T. \ 1997, \apj, 491, 2, 421
    \bibitem[Fang et al.(2018)]{Fang2018} Fang, F.; Forero-Romero, J.; Rossi, G.; Li, X. \& Feng, L\ 2018, arXiv, 1809.00438 astro-ph
    \bibitem[Leclercq(2015)]{2015JCAP...03..047L} Leclercq, F.; Jasche, J. et al.\ 2015, \jcap, 03, 047
    \bibitem[Schneider(2014)]{Schneider2014} Schneider, P. \ 2014, ``Extragalactic Astronomy and Cosmology'', Springer.
    \bibitem[Sutter et al.(2015)]{Sutter2015} Sutter, P. M. \& Lavaux G. \& Hamaus, N \& et al. \ 2015, A\&C, 9, 1-9
    \bibitem[van de Weygaert(2014)]{2016IAUS..308..493V} van de Weygaert, Rien\ 2014, Proceedings of the IAU, 308, 493   
  
  \end{thebibliography}                                                           
                       

%\listofchanges

\end{document}
