\documentclass[preprint]{aastex62}
% \documentclass[manuscript]{aastex62}
%%  twocolumn, manuscript, preprint, preprint, modern and RNAAS
\usepackage[utf8]{inputenc}
%\usepackage{siunitx}
%\usepackage[spanish]{babel}
%
\newcommand{\vdag}{(v)^\dagger}
\newcommand\aastex{AAS\TeX}
\newcommand\latex{La\TeX}
%% Tells LaTeX to search for image files in the 
%% current directory as well as in the figures/ folder.
\graphicspath{{./}{figures/}}
%% Reintroduced the \received and \accepted commands from AASTeX v5.2
%%\received{January 1, 2018}
%%\revised{January 7, 2018}
%%\accepted{\today}
\shorttitle{A LSS Void Identifier based on $\beta$-Skeleton}
\shortauthors{F. L. Gómez-Cortés}
  
\begin{document}

\title{A Large Scale Structure Void Identifier for Galaxy Surveys
  Based on the $\beta$-Skeleton Graph Method}

\correspondingauthor{Felipe Leonardo Gómez-Cortés}
\email{fl.gomez10@uniandes.edu.co}

\author{Felipe Leonardo Gómez-Cortés}
\affiliation{Physics Department, Universidad de Los Andes}
\collaboration{Master Student}
\collaboration{Code 201324084}

\nocollaboration


\author{Jaime E. Forero-Romero}
\affiliation{Physics Department, Universidad de Los Andes}
\collaboration{Advisor}


%%\begin{abstract}
\section*{Abstract}

  We are living the golden age of observational cosmology. 
  There is a consolidated standard cosmological model ($\Lambda$CDM) that explains the observed
  Large Scale Structure (LSS) of galaxies by introducing dark matter and
  dark energy as the dominant Universe components along with baryonic matter.
  Furthermore, we are able to do precise observatioanl measurements of the 
  cosmological parameters in that model. 
  Most of this success is due to computational cosmology that is now 
  an stablished tool to probe theoretical models and compare them with observations.
  The main features of the LSS can been reproduced in large cosmological N-body simulations.
  
  One of the most striking features in the LSS are voids; irregular 
  volumes on the order of tens of Mpc scales, where the matter density is below the Universe
  average density. 
  Statistics about voids such as its volume, shape and orientation also encode cosmological information.
  For this reason there is a great interest in algorithms that find and characterize voids
  both in simulations and observations.

  The objetive of this work is to develop a new void finder based
  on the $\beta$-Skeleton method.
  The $\beta$-Skeleton has been widely used on image processing,
  recognition and machine learnig applications, it has been introduced
  recently in LSS analysis. It is a fast tool identifiying LSS filamentary structure,
  and promises to be a robust tool to make cosmological tests.
  After developing the void finder we will characterize the $\beta$-skeleton voids
  in simulations and observations. We will also make prediction for the upcoming
  Dark Energy Spectroscopic Instrument about the void population that could be detected
  with that experiment.
  

  \medskip

  Keywords: Large Scale Structure, cosmology, voids, computational astrophysics

  \keywords{ Large Scale Structure, cosmology, voids, computational astrophysics}
  
%%\end{abstract}

  \section*{Resumen}

  Estamos viviendo en la era dorada de la cosmolog\'ia observacional.
  Existe un modelo est\'andar comol\'ogico ($\Lambda$-CDM) consolidado que explica
  las observaciones de la Estructura de Gran Escala (LSS) de galaxias mediante
  la introducci\'on de materia oscura y energ\'ia oscura como las componentes
  dominantes del Universo junto con la materia bari\'onica. M\'as a\'un, somos capaces de
  realizar mediciones precisas de los par\'ametros cosmol\'ogicos de este modelo a partir de
  observaciones. Gran parte de estos alcances es debido a la cosmolog\'ia computacional
  que es ahora una herramienta fuertemente establecida para probar modelos te\'oricos y
  compararlos con las observaciones. Las caracter\'isticas principales de la LSS pueden
  ser reproducidas en grandes simulaciones cosmol\'ogicas de N-cuerpos.

  Una de las caracter\'isticas m\'as prominentes en la LSS son los vac\'ios: vol\'umenes
  irregulares de escalas del orden de decenas de Mpc, donde la densidad de materia est\'a
  por debajo de la densidad media en el Universo. El an\'alisis estad\'istico de propiedades
  de los vac\'ios, como su volumen, forma y orientaci\'on tambi\'en nos puede dar informaci\'on
  cosmol\'ogica. Por esta raz\'on existe un gran inter\'es en algoritmos que encuentren y
  caractericen vac\'ios tanto en simulaciones como en observaciones.

  El objetivo de este trabajo es desarrollar un nuevo buscador de vac\'ios basado en el
  m\'etodo $\beta$-Skeleton. El m\'etodo $\beta$-Skeleton ha sido ampliamente utilizado
  en reconocimiento, procesamiento de im\'agenes y aplicaciones de \textit{machine learning},
  recientemente ha sido introducido en el an\'alisis de LSS. Esta es una herramienta r\'apida
  para identificar estructuras filamentarias en la LSS, y promete ser una herramienta robusta
  para realizar an\'alisis cosmol\'ogicos. Luego de desarrollar el buscador de vac\'ios
  caracterizaremos los vac\'ios del $\beta$-Skeleton en simulaciones y observaciones. Tambi\'en
  realizaremos predicciones para el experimento en desarrollo Dark Energy Spectroscopic
  Instrument (DESI) acerca de la poblaci\'on que podr\'a detectar.

  \medskip

  Palabras clave: estructura de gran escala, cosmología, vacíos,
  astrofísica computacional.


  \section{Introducción}

  La cosmolog\'ia f\'isica actual considera al Universo como un ente
  dinámico.  
  Hay dos aspectos principales en esta evoluci\'on: la expansi\'on que
  hace que la densidad global de materia disminuye
  con el tiempo cosmol\'ogico y, el segundo, la formaci\'on de
  estructuras localmente cada vez m\'as densas debido al colapso
  gravitacional. 

  Las observaciones del fondo de radiaci\'on c\'osmica de microondas
  ({\bf referencia}) y de la distribuci\'on de galaxias a gran escala
  ({\bf referencias}) apuntan a que esta evoluci\'on puede ser descrita
  por un pu\~nado de par\'ametros cosmol\'ogicos, donde los m\'as
  importantes son la densidad de materia y la densidad de energ\'ia
  oscura. 
  
  El reto de la cosmolog\'ia actual
 es aumentar la precisi\'on de las
  mediciones de estos par\'ametros cosmol\'ogicos. 
  Esto no s\'olo se logra con mediciones m\'as precisas sino con
  m\'etodos independientes para acotar los par\'ametros
  cosmol\'ogicos.
  Aunque un m\'etodo independiente pueda tener una incertidumbre
  grande, considerar las cotas impuestas por varios m\'etodos
  simult\'aneamente reduce la incertidumbre sobre los par\'ametros
  cosmol\'ogicos.

  Una de las pruebas cosmol\'ogicas que ha emergido en la \'ultima
  d\'ecada es la caracterizaci\'on de los vac\'ios cosmol\'ogicos;
  grandes regiones del espacio que cuentan con una baja densidad de
  galaxias. 

  

  \section{Estado del Arte}

Los vac\'ios cosmol\'ogicos son evidentes en mapas de la
distribuci\'on tridimensional de galaxias hechos en las \'ultimas dos
d\'ecadas \citep{SDSS-DR14-2017}.
El proyecto de mapeo más emblem\'atico de la primera década del
siglo XXI fue el \textit{Sloan Digital Sky Survey (SDSS)} el cu\'al
ha estimado la posici\'on en el espacio de más de 1.5 milones de galaxias de redshift

El trabajo de interpretaci\'on de estos mapas en t\'erminos de
par\'ametros cosmol\'ogicos requiere la realizaci\'on de simulaciones
que sigan la formaci\'on de estructuras en un inverso en expansi\'on.
Una simulación m\'as emblem\'atica en este aspecto es el Millenium Run
hecha en el Instituto Max Planck de Astrof\'isica hace 15 a\~nos. 

La Figura \ref{fig:pie_millenium_walls} compara los mapas de la
distribuci\'on de galaxias obtenidos por simulaciones (rojo) y por
observaciones (azul).
Cada punto representa una galaxia.
En estas im\'agenes los grandes filamentos donde se aglomeran la
galaxia se complementan con las regiones donde hay menos galxias que
conforman los vac\'ios que nos interesan en esta tesis.

  \begin{figure}
    \plotone{pie_millennium_walls}
    \caption{Distribuci\'on espacial de galaxias observada en mapeos
      como el SDSS y el 2dFGS (en azul) comparadas con resultados de
      la simulación Millenium (rojo). Max Planck Institute for
      Astrophysics.
      Cada punto representa una galaxia. 
      \label{fig:pie_millenium_walls}}
  \end{figure}
  
  DESI es un experimento que actualmente se encuentra en etapa de construcción. Usa un telescopio de 4m con
  dispositivos de ubicación automática de fibras para realizar espectroscopía. Puede ser considerado una versión
  potenciada del SDSS. 
{\bf (Completar un parrafo mas sobre DESI)}

  \section{Marco Teórico}

  Actualmente se dispone tanto de catálogos de halos de materia oscura de simulaciones como de catálogos de
  galaxias de observaciones.


  El método $\beta$-Skeleton ha sido introducido en astrofísica recientemente \citep{Fang2018} para
  encontrar el grafo de la LSS en catálogos de halos de materia oscura de distintas simulaciones.
  Ha probado ser una herramienta útil para detectar estructuras subyacentes diferentes en
  simulaciones cosmológicas generadas por métodos distintos.


 % \url{Testing cosmic geometry without dynamic distortions using voids}
 % \url{Hamaus, N., Sutter, P. M., Lavaux, G., & Wandelt, B. D. "Testing cosmic geometry without dynamic distortions using voids". JCAP 12, 013, December (2014). https://arxiv.org/pdf/1409.3580.pdf}


  Al estudiar la LSS se encuentra que la mayor parte del volúmen del Universo está dominado
  por los vacíos de la red cósmica. Pero por el desarrollo histórico de la astrofísica, se ha
  prestado más atención a las galaxias como marcadores de la estructura compuesta por materia
  bariónica y materia oscura. No en todos los halos de materia oscura se forman galaxias, así
  que hay un cierto sesgo (\textit{bias}) al tomar las galaxias como marcadores de los
  halos de materia oscura. Se han desarrollado numerosas pruebas estadísticas para estudiar
  estos marcadores, entre ellas las funciones de correlación galaxia-galaxia.

  Al construir los mapas tridimensionales del universo se convierten los observables (posición
  angular y corrimiento al rojo) en coordenadas cartesianas en el marco de referencia comovil.
  Para hacer esta conversión es necesario conocer la métrica de nuestro Universo; cómo se
  expande el espacio en función del tiempo. El modelo cosmológico estándar $\Lambda$CDM
  describe esta rata de expansión en función de algunos parámetros como la densidad de materia
  y la densidad de energía oscura en el Universo. Estos datos se pueden obtener con precisión
  de sondas como Planck y otro tipo de experimentos (BAO por ejemplo).

  Si se construye un mapa del universo con los parámetros cosmológicos erroneos, se observarán
  distorsiones en la distribución de los objetos. Este es el principio del test de
  \citet{AlcockPaczynski1979}.

  Al estudiar los vacíos de la red cósmica en simulaciones se encuentra estadísticamente una
  forma esferoidal. Se quieren obtener resultados similares al convertir observaciones de
  vacíos (a partir de las observaciones de galaxias como marcadores de sus límites) a coordenadas
  comóviles. Una descripción detallada de esta metodología se brinda en \citet{Hamaus2015}:

  Sea $x$ el vector que denota las coordenadas de un objeto en el espacio. La distancia comovil en la
  dirección de la línea de visión se calcula como:

  \begin{equation}
    x_{\parallel} = \int _0 ^z \frac{c}{H(z')}dz'
  \end{equation}

  Donde $c$ es la velocidad de la luz y $H(z)$ describe la expansión de Hubble como función del
  corrimento al rojo $z$. Si el objeto tiene velocidad radial, esta induce un corrimiento adicional
  hacia el rojo o hacia el azul. Al tener esto en cuenta queda:

  \begin{equation}
    \tilde{x}_{\parallel} = \int _0 ^{z + \frac{v_\parallel}{c}}(1+z) \frac{c}{H(z')}dz'
    \simeq x_\parallel + \frac{v_\parallel}{H(z)}(1+z)
  \end{equation}

  En tanto el ángulo $\theta$  entre un par de objetos objetos se relaciona con la distancia comovil según

  \begin{equation}
    x_{\perp} = D_A(z)\theta
  \end{equation}
  
  En una métrica de  Friedmann-Lema\^itre-Robertson-Walker de curvatura $k$, la distancia
  comovil angular viene dada por:
  
  \begin{equation}
    D_A(z) = \frac{c}{H_0\sqrt{-\Omega_k}} \sin \left( H_o\sqrt{-\Omega_k}
    \int_0^z \frac{1}{H(z'} dz' \right).
  \end{equation}
  
  Este término es bastante sensible al parámetro de curvatura
  $\Omega_k = 1 - \Omega_m - \Omega_\Lambda$, en tanto la rata de expansión de Hubble
  depende en sí misma del contenido de energía y materia en el Universo como:

  \begin{equation}
    H(z) = H_0 \sqrt{ \Omega_m(1+z)^3 + \Omega_k(1+z)^2+\Omega_\Lambda}
  \end{equation}

  De este modo se puede ver cómo es necesario asumir los valores actuales correctos de los
  parámetros cosmológicos para generar mapas tridimensionales de los objetos en el cielo. Si
  estos valores no coinciden con los parámetros
  cosmológicos reales, las distancias calculadas  en la lína de visión y perpendicular a ella serán
  incorrectas. Una de las herramientas fundamentales para realizar tests AP son las funciones
  de correlación. Estas pueden ser aplicadas entre pares de galaxias-vacíos o funciones de
  autocorrelación vacío-vacío.
  

  
  \section{Objetivos}

  \subsection{Objetivo Principal}
  Desarrollar un nuevo buscador de vacíos de la red cósmica (LSS) en catálogos de galaxias
  basado en el método $\beta$-Skeleton.
  
  \subsection{Objetivos Específicos}

  \begin{itemize}

      \item Identificar y catalogar vacíos de la red cósmica en catálogos de halos de materia oscura de
  simulaciones y en catálogos de galaxias desde observaciones.

      \item Calcular el parámetro cosmológico asociado a la energía oscura a partir del catálogo de
  vacíos de la red cósmica

    \item  Estimar los posibles resultados observacionales que pueda medir el experimento DESI.
  \end{itemize}
  
  \section{Metodología}

  La primera parte de este trabajo consiste en desarrollar el código del buscador de vacíos
  en la red cósmica. Aquí se incluye una etapa de calibración de parámetros libres del buscador.
  En la segunda parte se utilizará el código para obtener catálogos
  de vacíos de la red a partir de catálogos de halos de materia oscura de simulaciones o
  catálogos de galaxias de observaciones, mediante análisis estadístico de los catálogos de
  vacíos se compararán simulaciones y observaciones.
  Se calculan propiedades de las regiones como tamaño, forma, orientación. \citep{El-Ad1997}.
  Estas pueden ser comparadas con propiedades similares de catálogos generados por otros códigos
  como VIDE \citep{Sutter2015}.

  En la tercera parte se hará un análisis estadístico para determinar parámetros cosmológicos.
  Se espera encontrar estadísticamente que no hayan direcciones privilegiadas para la orientación
  de los ejes (asumiendo formas elipsoidales), funciones de correlación en distancias transversales
  y longitudinales a la línea de observación para hacer pruebas de \citet{AlcockPaczynski1979}.
  
  Finalmente se podrá dar un estimado de las observaciones que podrá llegar a medir el
  experimento DESI.

  \subsection{Desarrollo del código}

  El código será escrito en Python3 o C++, podrá ejecutarse en el Colaboratorio o en el HPC de la
  facultad de ciencas de la Universidad de los Andes.
  
  Para desarrollar el código se parte de la lectura de archivos estándar de catálogos de halos
  de materia oscura o distribución de galaxias. Estos puntos se ubican en un espacio tridimensional.

  Se dispone de la librería ``NGL''\citep{ngl} (Neighborhood Graph Library)
  para calcular la estructura $\beta$-Skeleton. Esta librería es de uso libre, está escrita en C++.
  Inicialmente fue desarrollada para estudiar topología de conjuntos de datos con un número de muestra
  pequeño. Puede encontrar vecinos en los conjuntos de puntos usando distintos métodos como los
  Grafos de Gabriel y el $\beta$-Skeleton, estos métodos resultan computacionalmente más económicos
  y rápidos que otros métodos más robustos como la trianguación de Delaunay.
  
  Por definición los vacíos cósmicos son regiones con baja densidad de materia, así que se
  estudiarán regiones de un campo escalar. Para esto es conveniente dividir el espacio en
  celdas discretas. El tamaño de las celdas será el primer parámetro a calibrar en el código.
  
  Se aplica el método $\beta$-Skeleton para conectar los puntos en el espacio y trazar los
  filamentos de la red cósmica. En este punto se utilizará la librería NGL \citep{ngl}.
  El parámetro $\beta$ de este método es otro parámetro a calibrar en el código. Se puede
  utilizar como guía el resultado obtenido por \citet{Fang2018}.
  Se transforman estos puntos y filamentos en un campo de densidad de materia. Este primer
  campo escalar de materia es bastante discreto y discontínuo. Será suavizado mediante un
  kernel Gaussiano, el número de veces que sea suavizado el campo de densidad y el tamaño de
  las celdas del kernel serán parámetros a calibrar en el código.

  Una vez suavizado el campo de densidad de materia se identifican los centroides
  de los vacíos de la red. Esto se puede realizar revisando, por ejemplo, puntos intermedios
  en conexiones largas que pueden aparecer en un $\beta$-skeleton con parámetro $\beta$ bajo,
  pero no aparecen en la red obtenida con el valor de $\beta$ apropiado.

  Ya identificados los centroides de los vacíos, se utiliza el método \textit{watershed}
  para identificar las regiones de baja densidad \citep{Sutter2015}. Con este método
  se planta una semilla en los centroides, todas las semillas crecen a la misma velocidad; una
  celda vecina a la vez. Se establecen ciertas reglas, por ejemplo las regiones no crecerán en
  las regiones donde el campo escalar de densidad tenga un valor alto, o se detendrán cuando
  se encuentren con otra región en crecimiento. Otro método a probar para identificar regiones
  es la construcción de esferas dentro de los vacíos de la red. 

  \section{Cronograma}

  \begin{table}[htb]
    \begin{tabular}{|c|cccccccccccccccc| }
      \hline
      Tareas $\backslash$ Semanas & 1 & 2 & 3 & 4 & 5 & 6 & 7 & 8 & 9 & 10 & 11 & 12 & 13 & 14 & 15 & 16  \\
      \hline
      1 & X & X & X & X & X & X &   &   &   &   &   &   &   &   &   &   \\
      2 &   &   &   &   & X & X &   &   &   &   &   &   &   &   &   &   \\
      3 &   &   &   &   &   &   & X & X &   &   &   &   &   &   &   &   \\
      4 &   &   &   &   &   &   & X & X & X &   &   &   &   &   &   &   \\
      5 &   &   &   &   &   &   &   &   & X & X &   &   &   &   &   &   \\
      6 &   &   &   &   &   &   &   &   &   &   & X & X & X &   &   &   \\
      7 &   &   &   &   &   &   &   &   &   &   &   &   &   & X & X & X \\
      8 & X & X & X & X & X & X & X &   &   &   &   &   &   &   &   &   \\
      9 &   &   &   &   &   &   & X & X & X &   &   &   &   &   &   &   \\
      10&   &   &   &   &   &   &   &   &   & X & X & X & X &   &   &   \\
      11&   &   &   &   &   &   &   &   &   &   &   &   &   & X & X & X \\
      12&   &   &   &   &   &   &   &   &   &   &   &   &   &   &   & X \\
      \hline
    \end{tabular}
  \end{table}

  
  \begin{enumerate}
  \item Desarrollo del código del Buscador de Vacíos Basado en $\beta$-Skeleton.
  \item Calibración de los parámetros del código y comparación con otros buscadores.
  \item Obtención de catálogos de vacíos de la red a partir de simulaciones.
  \item Obtención de catálogos de vacíos de la red a partir de observaciones.
  \item Comparación por análisis estadístico entre simulaciones y observaciones.
  \item Cálculo de la constante cosmológica a partir de catálogo de vacíos de la red en
    observaciones.
  \item Estimación de resultados para el experimento DESI.
  \item Escritura del Documento: Introducción y marco teórico.
  \item Escritura del Documento: Desarrollo y calibración del Código.
  \item Escritura del Documento: Análisis de catálogos de vacíos de la red cósmica. 
  \item Escritura del Documento: Estimación de posibles resultados del experimento DESI.
  \item Escritura del Documento: Conclusiones
  \end{enumerate}
  
  
  \section{Resultados Esperados}

  Como punto de referencia se tiene el catálogo de vacíos generado por el código VIDE \citep{Sutter2015}.
  
%  \url{  https://arxiv.org/pdf/1204.5185.pdf} Void Galaxies from SDSS
%  \url{  https://arxiv.org/abs/0711.2480 } Alignments of voids in the cosmic web
    
  \nocite{*}

  \begin{thebibliography}{}
    \bibitem[Aarseth(2003)]{Aarseth2003} Aarseth, S. J. \ 2003, ``Gravitational N-Body Simulations'', Cambridge University Press.
    \bibitem[Alcock-Paczynski(1979)]{AlcockPaczynski1979} Alcock, C. \& Paczy\'nski, B.\ 1979, \nat, 281, 358    
    \bibitem[Correa \& Lindstrom(2012)]{ngl} Correa, Carlos \& Lindstrom, Peter.\ 2011,  IEEE TVCG\ 17,12 (Dec 2011), 1852-1861
    \bibitem[El-Ad \& Piran(1997)]{El-Ad1997} El-Ad, H. \& Piran, T. \ 1997, \apj, 491, 2, 421
    \bibitem[Fang et al.(2018)]{Fang2018} Fang, F.; Forero-Romero, J.; Rossi, G.; Li, X. \& Feng, L\ 2018, arXiv, 1809.00438 astro-ph
    \bibitem[Hamaus et al.(2015)]{Hamaus2015} Hamaus, N.; Sutter, P.M.; Lavaux, G. \& Wandelt, B. D. \ 2015, \jcap, 11, 036    
    \bibitem[Leclercq et al.(2015)]{2015JCAP...03..047L} Leclercq, F.; Jasche, J. et al.\ 2015, \jcap, 03, 047
    \bibitem[Longair(2004)]{Longair2004} Longair, M. S. \ 2004, ``A Brief History of Cosmology'', ``Measuring and Modeling the Universe'', Carnegie Observatories Astrophysics Series, Vol 2.
    \bibitem[Press \& Schechter(1974)]{Press&Schechter1974} Press, W. H. \& Schechter, P. \ 1974, \apj, 187, 425-438
    \bibitem[Schneider(2014)]{Schneider2014} Schneider, P. \ 2014, ``Extragalactic Astronomy and Cosmology'', Springer
    \bibitem[Springel et al.(2005)]{Springel2005} Springel, V. et al. \ 2005, \nat, 435, 639
    \bibitem[Sutter et al.(2015)]{Sutter2015} Sutter, P. M.; Lavaux G.; Hamaus, N; et al. \ 2015, A\&C, 9, 1-9
    \bibitem[van de Weygaert(2014)]{2016IAUS..308..493V} van de Weygaert, Rien\ 2014, Proceedings of the IAU, 308, 493      

    \bibitem[SDSS Collaboration(2017)]{SDSS-DR14-2017} SDSS Collaboration \ 2017, arxiv, 1707.09322 astro-ph
    \bibitem[Hinshaw et al.(2013)]{WMAP2013} Hinshaw, G. et al. \ 2013, \apjs, 208, 20.
  \end{thebibliography}                                                           
                       

%\listofchanges

\end{document}
