\documentclass[manuscript]{aastex62}
% \documentclass[manuscript]{aastex62}
%%  twocolumn, manuscript, preprint, preprint, modern and RNAAS
\newcommand{\vdag}{(v)^\dagger}
\newcommand\aastex{AAS\TeX}
\newcommand\latex{La\TeX}
%% Tells LaTeX to search for image files in the 
%% current directory as well as in the figures/ folder.
\graphicspath{{./}{figures/}}
%% Reintroduced the \received and \accepted commands from AASTeX v5.2
%%\received{January 1, 2018}
%%\revised{January 7, 2018}
%%\accepted{\today}
\shorttitle{A LSS Void Identifier based on $\beta$-Skeleton}
\shortauthors{F. L. G\'omez-Cort\'es}


\begin{document}

\title{A Large Scale Structure Void Identifier for Galaxy Surveys
  Based on the $\beta$-Skeleton Graph Method}

\correspondingauthor{Felipe Leonardo G\'omez-Cort\'es}
\email{fl.gomez10@uniandes.edu.co}

\author{Felipe Leonardo G\'omez-Cort\'es}
\affiliation{Physics Department, Universidad de Los Andes}
\collaboration{Master Student}
\collaboration{Code 201324084}

\nocollaboration


\author{Jaime E. Forero-Romero}
\affiliation{Physics Department, Universidad de Los Andes}
\collaboration{Advisor}


%%\begin{abstract}
\section*{Abstract}

  We are living the golden age of observational cosmology. 
  There is a consolidated standard cosmological model ($\Lambda$CDM) that explains the observed
  Large Scale Structure (LSS) of galaxies by introducing dark matter and
  dark energy as the dominant Universe components along with baryonic matter.
  Furthermore,  we are able to do precise observatioanl measurements of the 
  cosmological parameters in that model. 
  Most of this success is due to computational cosmology that is now 
  an stablished tool to probe theoretical models and compare them with observations.
  The main features of the LSS can been reproduced in large cosmological N-body simulations
  such as Millenium and Bolshoi. 
  
  One of the most striking features in the LSS are voids; irregular 
  volumes on the order of tens of Mpc scales, where the matter density is below the Universe
  average density. 
  Statistics about voids population such as
  mass, shape and orientation also encode cosmological information, for instance, about Dark Energy.
  For this reason there is great interest in algorithms that find and characterize voids
  both in simulations and observations.

  The objetive of this work is to develope a new void finder  based
  on the $\beta$-Skeleton method.
  The $\beta$-Skeleton has been widely used on image processing,
  recognition and machine learnig applications, has been introduced
  recently in LSS analysis. It is a fast tool identifiying LSS filamentary structure,
  and promises to be a robust tool to make cosmological tests.
  
  After developing the void finder we will characterize the $\beta$-skeleton voids
  in simulations and observations. We will also make prediction for the upcoming
  Dark Energy Spectroscopic Instrument about the void population that could be detected
  with that experiment.
  

  

  \medskip

  Keywords: Large Scale Structure, cosmology, voids, computational astrophysics

  \keywords{ Large Scale Structure, cosmology, voids, computational astrophysics}
  
%%\end{abstract}

  \section*{Resumen}

  Estamos viviendo en la era dorada de la cosmolog\'ia observacional.
  Desde las \'ultimas tres d\'ecadas podemos realizar mediciones precisas
  de los par\'ametros cosmol\'ogicos. Adem\'as,
  las astrof\'isica computacional se ha establecido como una herramienta para probar
  modelos te\'oricos y compararlos con las observaciones.

  El modelo cosmol\'ogico estandar ($\Lambda$CDM) explica la Estructura de Gran
  Escala (LSS) observada en las galaxias mediante la introducci\'on de la materia
  oscura y la energ\'ia oscura como los componentes dominantes del universo
  junto con la materia bari\'onica. La LSS puede ser reproducida en grandes
  simulaciones cosmol\'ogicas de N-cuerpos como Millenium y Bolshoi.

  Una de las caracter\'isticas m\'as destacadas de la LSS son los vac\'ios:
  vol\'umenes irregulares de escalas del orden de decenas de Mpc, donde la
  densidad de materia est\'a por debajo del $20\%$ del promedio de la densidad
  media del universo y se escasamente se encuentran galaxias de baja masa.
  Los vac\'ios en la LSS pueden ser usados para estudiar la energ\'ia oscura y pueden dar
  pistas importantes sobre otros par\'ametros comol\'ogicos.
  La estad\'istica sobre la poblaci\'on de vac\'ios, como la masa,
  forma y orientaci\'on, guardan esa informaci\'on.

  El m\'etodo $\beta$-Skeleton ha sido ampliamente utilizado en procesamiento
  de im\'agenes, reconocimiento y aplicaciones de \textit{machine learning},
  recientemente ha sido introducido en el an\'alisis de la LSS.
  Es una herramienta r\'apida para la identificaci\'on de los filamentos
  en la LSS, promete ser una herramienta robusta para realizar an\'alisis
  estad\'istico como la funci\'on de correlaci\'on de dos puntos y
  la prueba de Alcock-Pazcy\'nski, ambos  m\'etodos usados actualmente en
  cosmolog\'ia.

  El objetivo de este trabajo es desarrollar un identificador de vac\'ios
  de la LSS basados en el m\'etodo $\beta$-Skeleton para mejorar el an\'alisis
  estad\'istico de los vac\'ios de la LSS en cat\'alogos de galaxias
  y la restricci\'on de par\'ametros cosmol\'ogicos.

  \medskip

  Palabras Clave: estructura de gran escala, cosmolog\'ia, vac\'ios,
  astrof\'isica computacional.

\nocite{*}


\begin{thebibliography}{}                                                       

\bibitem[van de Weygaert(2014)]{2016IAUS..308..493V} van de Weygaert, Rien\ 2014, Proceedings of the IAU, 308, 493   
\bibitem[Fang(2018)]{arXiv:1809.00438} Fang, F.; Forero-Romero, J.; Rossi, G.; Li, X. \& Feng, L\ 2018, arXiv, 1809.00438 astro-ph
\bibitem[Alcock-Paczynski(1979)]{1979Natur.281..358A} Alcock, C. \& Paczy\'nski, B.\ 1979, \nat, 281, 358
\bibitem[Leclercq(2015)]{2015JCAP...03..047L} Leclercq, F.; Jasche, J. et al.\ 2015, \jcap, 03, 047
                                                                                
\end{thebibliography}                                                           
                       

%\listofchanges

\end{document}
